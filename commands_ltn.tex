%----------------------------------------------------------------%
%-------------------- Shortcut commands -------------------------%
%----------------------------------------------------------------%
\newcommand{\Dfrac}[2]{\ensuremath \frac{\mbox{d} #1}{\mbox{d} #2}}
\newcommand{\dint}{\ensuremath \,\mathrm{d}}
\newcommand{\etal}{{\em et al}}
\newcommand{\Ys}{\ensuremath Y^*}
\newcommand{\Ts}{\ensuremath T^*}
\newcommand{\cs}{\ensuremath c^*}
\newcommand{\zs}{\ensuremath z^*}
\newcommand{\Zs}{\ensuremath Z^*}
\newcommand{\dYs}{\mathrm{d} Y^*}
\newcommand{\dTs}{\mathrm{d} T^*}
\newcommand{\dzs}{\mathrm{d} z^*}
\newcommand{\dZs}{\mathrm{d} Z^*}
\newcommand{\dcs}{\mathrm{d} c^*}
\newcommand{\dzstar}{\mathrm{d} z\star}                                   
\newcommand{\dcstar}{\mathrm{d} c\star}
\newcommand{\rrate}{\ensuremath {\dot{\omega}}}
\newcommand{\todo}[1]{{TODO: \textit{\small #1}\\}}


% Shortcut commands
\newcommand{\Matrix}[1]{\ensuremath \mathsf{#1}}
\newcommand{\Vector}[1]{\ensuremath \mathbf{#1}}
\newcommand{\Tensor}[1]{\ensuremath \vec{\vec{#1}}}
\newcommand{\Pfrac}[2]{\ensuremath \frac{\partial #1}{\partial #2}}
\newcommand{\pf}[2]{\ensuremath \frac{\partial #1}{\partial #2}}
\newcommand{\Rfrac}[2]{\ensuremath \frac{\mbox{d} #1}{\mbox{d} #2}}
\newcommand{\Dvolume}{\ensuremath {d\Omega}}
\newcommand{\Dsurface}{\ensuremath dS}
\newcommand{\Intvolume}[1]{\ensuremath \int_\Omega {#1}\Dvolume}
\newcommand{\Intsurface}[1]{\ensuremath \int_{\Dvolume} {#1}\Dsurface}
\newcommand{\Intsurfacedot}[1]{\ensuremath \int_{\Dvolume} {#1}\cdot\Vector{\Dsurface}}
\newcommand{\Div}[1]{\ensuremath \Vector{\nabla}\cdot#1}
\newcommand{\Grad}[1]{\ensuremath \Vector{\nabla #1}}
\newcommand{\Sumedges}{\ensuremath \sum_{ik}^{\text{edges}}}
\newcommand{\nutilde}{\ensuremath \tilde{\nu}}
\newcommand{\Dx}{\ensuremath \Delta x}
\newcommand{\Dy}{\ensuremath \Delta y}
\newcommand{\Ds}{\ensuremath \Delta s}
\newcommand{\Dl}{\ensuremath \Delta l}
\newcommand{\Dt}{\ensuremath \Delta t}
\newcommand{\Tilda}[1]{\ensuremath \mbox{\~{#1}}}
\newcommand{\pseudo}{\ensuremath \textrm{\textdied}}
%\newcommand{\Fluxi}{\overrightarrow{\ensuremath{\mathbf{F}_{\boldsymbol{\iota}}} }}
\newcommand{\Fluxi}{\overrightarrow{\ensuremath{\mathbf{F}} }}
\newcommand{\Fluxv}{\overrightarrow{\ensuremath{\mathbf{F_{v}}} }}
%\newcommand{\Fi}{\ensuremath{\mathbf{F}_{\boldsymbol{\iota}}} }
\newcommand{\Fi}{\ensuremath{\mathbf{F}} }
%\newcommand{\Fv}{\ensuremath{\mathbf{F}_{\boldsymbol{\nu}}}}
\newcommand{\Fv}{\ensuremath{\mathbf{F_{v}}} }
%\newcommand{\Gi}{\ensuremath{\mathbf{G}_{\boldsymbol{\iota}}} }
\newcommand{\Gi}{\ensuremath{\mathbf{G}} }
%\newcommand{\Gv}{\ensuremath{\mathbf{G}_{\boldsymbol{\nu}}} }
\newcommand{\Gv}{\ensuremath{\mathbf{G_{v}}} }
%\newcommand{\Hi}{\ensuremath{\mathbf{H}_{\boldsymbol{\iota}}} }
\newcommand{\Hi}{\ensuremath{\mathbf{H}} }
%\newcommand{\Hv}{\ensuremath{\mathbf{H}_{\boldsymbol{\nu}}} }
\newcommand{\Hv}{\ensuremath{\mathbf{H_{v}}} }
\newcommand{\Uavg}{\ensuremath{\mathbf{\overline{U}}} }


% \encircle{width}{EllipseWidth}{EllipseHeight}{text}              % Requires \usepackage{pgf} (defined above amsmath)
\newcommand{\encircle}[4]{
\begin{pgfpicture}{- #1 cm}{-0.115 cm}{#1 cm}{0cm} 
\pgfputat{\pgfxy(0,0)}{\pgfbox[center,center]{#4}} 
\pgfellipse[stroke]{\pgfxy(0,0)}{\pgfxy(#2,0)}{\pgfxy(0,#3)}
\end{pgfpicture}
}
\newcommand{\mathencircle}[4]{
\begin{pgfpicture}{- #1 cm}{-0.05 cm}{#1 cm}{0cm} 
\pgfputat{\pgfxy(0,0)}{\pgfbox[center,center]{$ #4 $}} 
\pgfellipse[stroke]{\pgfxy(0,0)}{\pgfxy(#2,0)}{\pgfxy(0,#3)}
\end{pgfpicture}
}

%%%%%%%%%%%%%%%%%%%%%%%%%%%%%%%%%%%%%%%%%%%%%%%%%%%%%%%%%%%%%%%%%%%%%%%%%%%%%%
% Redefine some of the nomenclature commands
%\renewcommand{\nomname}{List of Symbols}
%\newcommand{\nomalpha}[2]{\nomenclature[ar ]{#1}{#2}}
%\newcommand{\nomgreek}[2]{\nomenclature[bg ]{#1}{#2}}
%\newcommand{\nomsuper}[2]{\nomenclature[cp ]{#1}{#2}}
%\newcommand{\nomsub}[2]{\nomenclature[db ]{#1}{#2}}
%
%% change the look of the default nomenclature commands
%\renewcommand{\nomgroup}[1]{%
%  \ifthenelse{\equal{#1}{A}}{\item[\textbf{Alphanumeric Symbols}]}{%
%   \ifthenelse{\equal{#1}{B}}{\item[\textbf{Greek Symbols}]}{%
%     \ifthenelse{\equal{#1}{C}}{\item[\textbf{Superscripts}]}{%
%       \ifthenelse{\equal{#1}{D}}{\item[\textbf{Subscripts}]}{}{}}{}}{}}{}}
%

% Add a new column type in table environment (Requires \usepackage{array} )
% centered paragraph
\newcolumntype{C}[1]{>{\centering\arraybackslash}p{#1}}


% Alter some LaTeX defaults for better treatment of figures:
% See p.105 of "TeX Unbound" for suggested values.
% See pp. 199-200 of Lamport's "LaTeX" book for details.
% General parameters, for ALL pages:
\renewcommand{\topfraction}{0.9}	% max fraction of floats at top
\renewcommand{\bottomfraction}{0.8}	% max fraction of floats at bottom
% Parameters for TEXT pages (not float pages):
\setcounter{topnumber}{2}
\setcounter{bottomnumber}{2}
\setcounter{totalnumber}{4}     % 2 may work better
\setcounter{dbltopnumber}{2}    % for 2-column pages
\renewcommand{\dbltopfraction}{0.9}	% fit big float above 2-col. text
\renewcommand{\textfraction}{0.07}	% allow minimal text w. figs
% Parameters for FLOAT pages (not text pages):
\renewcommand{\floatpagefraction}{0.7}	% require fuller float pages
% N.B.: floatpagefraction MUST be less than topfraction !!
\renewcommand{\dblfloatpagefraction}{0.7}	% require fuller float pages


% Added by Ramy Rashad (courtesy of Lucian Ivan) 
% Modified by Sean McDonald

\newcommand {\al}{\hspace*{9mm}}                                %0
\newcommand {\p} {\partial}
\newcommand {\ds} {\displaystyle}
\newcommand {\vs} {\vspace*{2mm}}
\newcommand {\vsb} {\vspace*{5mm}}
\newcommand {\ns} {\hspace*{-5mm}}
\newcommand {\spp}[1] {#1_{p_{1}p_{2}}}
\newcommand {\Sp}[1] {#1^{p_{1}}}
\newcommand {\Spp}[1] {#1^{p_{2}}}
\newcommand {\Sppp}[1] {#1^{p_{3}}}
\newcommand {\alpp} {\alpha_{p_{1}p_{2}}}
\newcommand {\Dpp} {D_{p_{1}p_{2}}}
\newcommand {\Dppp} {D_{p_{1}p_{2}p_{3}}}
\newcommand {\Ipp} {I_{p_{1}p_{2}}}
\newcommand {\Sij}[1] {#1^{i,j}}
\newcommand {\sij}[1] {#1_{i,j}}
\newcommand {\Sijk}[1] {#1^{i,j,k}}
\newcommand {\sijk}[1] {#1_{i,j,k}}
\newcommand {\Sumpp} {\sum_{p_{1}=0}^{k}\sum_{p_{2}=0}^{k}}
\newcommand {\Sumppp} {\sum_{p_{1}=0}^{k}\sum_{p_{2}=0}^{k}\sum_{p_{3}=0}^{k}}
\newcommand {\SumpppZero}{\mathop{\sum_{p_{1}=0}^{k}\sum_{p_{2}=0}^{k}\sum_{p_{3}=0}^{k}}_{(p_{1}+p_{2}+p_{3} \leq k)}}
\newcommand {\SumpppOne}{\mathop{\sum_{p_{1}=0}^{k}\sum_{p_{2}=0}^{k}\sum_{p_{3}=0}^{k}}_{(p_{1}+p_{2}+p_{3} \neq 0)}}
\newcommand {\SC}[1] {#1^{cell}}
\newcommand {\sC}[1] {#1_{cell}}
\newcommand {\Rijk} {R_{j}^{k}(\vec{r},u_{*}^{i})}

\newcommand {\IIG} {\int\hspace{-2.5mm}\int\limits_{\!\!\!\mathcal{G}}}
\newcommand {\IIIG} {\int\hspace{-2.5mm}\int\hspace{-2.5mm}\int\limits_{\!\!\!\mathcal{G}}}
\newcommand {\IIIdomain}[1] {\int\hspace{-2.5mm}\int\hspace{-2.5mm}\int\limits_{\!\!\!{#1}}}
\newcommand {\IIGZ} {\int\hspace{-3.5mm}\int\limits_{\hspace{-2mm}
                                                    \mathcal{G(\zeta,\theta)}}}
\newcommand {\XYZhat}[4] {\left(\widehat{{x^{#1} y^{#2} z^{#3}}}\right)_{#4}}
\newcommand {\XYhat}[3] {\left(\widehat{{x^{#1} y^{#2} }}\right)_{#3}}
\newcommand {\XYZtilde}[4] {\left(\widetilde{{x^{#1} y^{#2} z^{#3}}}\right)_{#4}}
\newcommand {\XYZbar}[4] {\Big(\overline{{x^{#1} y^{#2} z^{#3}}}\Big)_{#4}}
\newcommand {\XYbar}[3] {\Big(\overline{{x^{#1} y^{#2}}}\Big)_{#3}}
\newcommand {\prodXY} {\Sp{\left(x-x_{i}\right)}
                       \Spp{\left(y-y_{i}\right)}}
\newcommand {\prodXYZ} {\Sp{\left(x-x_{i}\right)}
                       \Spp{\left(y-y_{i}\right)}
		       \Sppp{\left(z-z_{i}\right)}}
\newcommand {\sg}[1] {#1_{g}}
\newcommand {\Sg}[1] {#1^{g}}
\newcommand {\Sumg} {\sum\limits_{g=1}^{NN}}
\newcommand {\DiffU} {\sg{\bar{U}} - \sij{\bar{U}}}
\newcommand {\R} {\right}
\newcommand {\dx} {\Delta\sC{x}}
\newcommand {\dy} {\Delta\sC{y}}
\newcommand {\ddx} {\frac{\p}{\p x}}
\newcommand {\ddy} {\frac{\p}{\p y}}
\newcommand {\foo} {f(x_{0},y_{0})}

\newcommand {\Uxyz} {U_{ijk}^{k}(x,y,z)}
\newcommand {\Uexact} {U_{\text{exact}}(x,y,z)}

\newcommand {\Ui} {U_{i}^{k}(x,y,z)}
\newcommand {\Uj} {U_{i}^{k}(x,y,z)}
\newcommand {\Uiavg} {\overline{U_{i}} }
\newcommand {\Ujavg} {\overline{U_{j}} }

\newcommand {\XYZ} {\Sp{\left(x-x_{ijk}\right)}\Spp{\left(y-y_{ijk}\right)}\Sppp{\left(z-z_{ijk}\right)}}
\newcommand {\XY} {\Sp{\left(x-x_{ijk}\right)}\Spp{\left(y-y_{ijk}\right)}}
\newcommand {\XYZi} {\Sp{\left(x-x_{i}\right)}\Spp{\left(y-y_{i}\right)}\Sppp{\left(z-z_{i}\right)}}
\newcommand {\XYi} {\Sp{\left(x-x_{i}\right)}\Spp{\left(y-y_{i}\right)}}
\newcommand {\XYZj} {\Sp{\left(x-x_{j}\right)}\Spp{\left(y-y_{j}\right)}\Sppp{\left(z-z_{j}\right)}}
\newcommand {\XYj} {\Sp{\left(x-x_{j}\right)}\Spp{\left(y-y_{j}\right)}}
\newcommand {\XYZa} {\Sp{\left(x-x_{\alpha}\right)}\Spp{\left(y-y_{\alpha}\right)}\Sppp{\left(z-z_{\alpha}\right)}}
\newcommand {\XYZb} {\Sp{\left(x-x_{\beta}\right)}\Spp{\left(y-y_{\beta}\right)}\Sppp{\left(z-z_{\beta}\right)}}

\newcommand {\Vol} {\mathcal{V}}
% remember to use [htp] or [htpb] for placement

% Add some preamble to the nomenclature page
%\renewcommand{\nompreamble}{The following typesetting convention is
%  applied throughout this document: scalar $s$, vector $\Vector{v}$,
%  tensor $\Tensor{t}$, and matrix $\Matrix{m}$. No convention is
%  implied by case.}

%%%%%%%%%%%%%%%%%%%%%%%%%%%%%%%%%%%%%%%%%%%%%%%%%%%%%%%%%%%%%%%%%%%%%%%%%%%%%%%
% Provides some info for the pdf file version
%\makeatletter
%\special{pdf: docinfo 
%<< 
%/Author (\@author)
%/Title (\@title)
%/Keywords (parallel, diffusion-flames, finite-volume, AMR, combusion)
%>> }
%\makeatother
%
%\DeclareGraphicsRule{.pdf}{eps}{.bb}{}
%
%%%%%%%%%%%%%%%%% Predetermined graphic sizes %%%%%%%%%%%%%%%%%%%%%%%%%%%%%%%%%
%
%\newcommand{\incfig}[1]{%
%  \begin{center}%
%    \includegraphics*[totalheight=0.3\textheight,keepaspectratio="true",clip]{#1}%
%  \end{center}
%}
%
%\newcommand{\incsubfig}[2]{%
%  \subfigure[#2]{%
%    \includegraphics*[width=0.48\textwidth,keepaspectratio="true",clip]{#1}%
%  }%
%}
%







%%----------------------------------------------------------------%
%%--------------------  Style            -------------------------%
%%----------------------------------------------------------------%
%\titleformat*{\section}{\rmfamily\bfseries\Large}
%\titleformat*{\subsection}{\rmfamily\bfseries\large}
%\titleformat*{\subsubsection}{\rmfamily\bfseries}
%
%% chapter 
%\makeatletter
%\renewcommand*\@makechapterhead[1]{%
%  \vspace*{50\p@}%
%  {\parindent \z@  \raggedleft \normalfont  %\raggedleft \center  
%    \ifnum \c@secnumdepth >\m@ne
%        \huge \bfseries\rmfamily \@chapapp\space \thechapter
%        \par\nobreak
%        \vskip 20\p@
%    \fi
%    \interlinepenalty\@M
%    \hrule
%    \vspace*{10\p@}% 
%    \raggedright
%    \Huge \bfseries\rmfamily{#1}\par\nobreak  % \sffamily
%    \vspace*{10\p@}%
%    \hrule
%    \vskip 40\p@
%  }}
%\makeatother
%
%\makeatletter
%\renewcommand*\@makeschapterhead[1]{%
%  \vspace*{50\p@}%
%  {\parindent \z@  \raggedright \normalfont  %\raggedleft  \center  
%    \interlinepenalty\@M
%    \hrule
%    \vspace*{10\p@}% 
%    \Huge \bfseries\rmfamily{#1}\par\nobreak %\sffamily
%    \vspace*{10\p@}%  
%    \hrule
%    \vskip 40\p@
%  }}
%\makeatother
%
%
%% short titles in chaptermark
%\makeatletter
%\newcommand*\std@chapter{}
%\let\std@chapter=\@chapter
%\renewcommand*\@chapter[2][]{\std@chapter[#2]{#2}\chaptermark{#1}}
%\makeatother
%
%
%% treatment of figures
%\renewcommand{\topfraction}{0.9}    % max fraction of floats at top
%\renewcommand{\bottomfraction}{0.9} % max fraction of floats at bottom
%\renewcommand{\textfraction}{0.1}   % allow minimal text w. figs
%\renewcommand{\floatpagefraction}{0.9} % require fuller float pages
%
%
%
%
%
%%----------------------------------------------------------------%
%%--------------------  Nomenclature     -------------------------%
%%----------------------------------------------------------------%
%% Redefine some of the nomenclature commands
%\renewcommand{\nomname}{List of Symbols}
%\renewcommand{\bibsection}{\chapter*}
%
%\renewcommand{\nomgroup}[1]{%
%  \ifthenelse{\equal{#1}{A}}{\item[\textbf{Alphanumeric Symbols}]}{%
%   \ifthenelse{\equal{#1}{B}}{{\vskip 1em}\item[\textbf{Greek Symbols}]}{%
%     \ifthenelse{\equal{#1}{C}}{{\vskip 1em}\item[\textbf{Superscripts}]}{%
%       \ifthenelse{\equal{#1}{D}}{{\vskip 1em}\item[\textbf{Subscripts}]}{%
%          \ifthenelse{\equal{#1}{E}}{{\vskip 1em}\item[\textbf{Other Symbols}]}{%
%             \ifthenelse{\equal{#1}{F}}{{\vskip 1em}\item[\textbf{Abbreviations}]}{}{}}{}}{}}{}}{}}{}}
%
%% Add some preamble to the nomenclature page
%%\renewcommand{\nompreamble}{The following typesetting convention is
%%  applied throughout this document: scalar $s$, vector $\Vector{v}$,
%%  tensor $\Tensor{t}$, and matrix $\Matrix{m}$. No convention is
%%  implied by case.}
%
%
%%%%%%%%%%%%%%%%%%%%%%%%%%%%%%%%%%%%%%%%%%%%%%%%%%%%%%%%%%%%%%%%%%%%%%%%%%%%%%%%
%% Provides some info for the pdf file version
%%\makeatletter
%%\special{pdf: docinfo 
%%<< 
%%/Author (\@author)
%%/Title (\@title)
%%/Keywords (parallel, diffusion-flames, finite-volume, AMR, combusion)
%%>> }
%%\makeatother
%%
%%\DeclareGraphicsRule{.pdf}{eps}{.bb}{}
%%
%%%%%%%%%%%%%%%%%% Predetermined graphic sizes %%%%%%%%%%%%%%%%%%%%%%%%%%%%%%%%%
%%
%%\newcommand{\incfig}[1]{%
%%  \begin{center}%
%%    \includegraphics*[totalheight=0.3\textheight,keepaspectratio="true",clip]{#1}%
%%  \end{center}
%%}
%%
%%\newcommand{\incsubfig}[2]{%
%%  \subfigure[#2]{%
%%    \includegraphics*[width=0.48\textwidth,keepaspectratio="true",clip]{#1}%
%%  }%
%%}
%%
