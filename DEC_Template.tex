\documentclass{beamer}
\usetheme{PaloAlto}
\useinnertheme{rectangles}
\usecolortheme{beaver}
\usepackage[version=3]{mhchem}
\usepackage{caption}
\usepackage{textpos}
\usepackage{etoolbox}
%\usepackage[usenames,dvipsnames]{xcolor}

\setbeamercolor{section in sidebar}{fg=darkgray}            %color of the active section
\setbeamercolor{section in sidebar shaded}{fg=lightgray}     %color of the inactive section
\setbeamercolor{subsection in sidebar}{fg=darkred}         %color of the active subsection
\setbeamercolor{subsection in sidebar shaded}{fg=lightgray}  %color of the inactive subsection
%\setbeamercolor{title in sidebar}{fg=...}              %color of the presentation title
%\setbeamercolor{author in sidebar}{fg=...}             %color of the author   


% Define footer
\setbeamertemplate{footline}{bg=primary}
\makeatother
\setbeamertemplate{footline}
{
    \leavevmode%
    \hbox{%
    \begin{beamercolorbox}[wd=.4\paperwidth,ht=2.25ex,dp=1ex,center]{author in head/foot}%
       \usebeamerfont{author in head/foot}\insertshortauthor\ (Chris Ngigi - DEC I)
    \end{beamercolorbox}%
    
    \begin{beamercolorbox}[wd=.4\paperwidth,ht=2.25ex,dp=1ex,center]{title in head/foot}%
        \usebeamerfont{title in head/foot}\insertshorttitle
        \hfill
    \end{beamercolorbox}    
        
    \begin{beamercolorbox}[wd=.2\paperwidth,ht=2.25ex,dp=1ex,center]{title in head/foot}%
        \usebeamerfont{title in head/foot}
        \hfill    
        \hspace{2em}\insertframenumber/\inserttotalframenumber
       % \insertframenumber{}\hspace{1ex}
    \end{beamercolorbox}}%
}

\beamertemplatenavigationsymbolsempty  % disable navigation
% ==================================================================
 \setbeamertemplate{sidebar}{%
   \vspace*{0cm}
    \vspace*{\fill}
   \vspace*{0.1cm}
   \setbeamersize{sidebar width left=2cm}
   \includegraphics[width=0.55cm]{./figs/utias_logo_blue.jpg}
%   \includegraphics[width=1.25cm]{weiti_logo.png}\\
%   \includegraphics[width=1.25cm]{pw_logo.png}
%    \vfill 
}
    
\setbeamerfont{section number projected}{%
  family=\rmfamily,series=\bfseries,size=\small}
\setbeamercolor{section number projected}{bg=lightgray,fg=darkred}
\setbeamertemplate{sections/subsections in toc}[circle]

\addtobeamertemplate{frametitle}{}{%
\begin{textblock*}{1cm}(-1.8cm,-1.2cm)
\includegraphics[height=1cm,width=1cm]{./figs/utias-logo2.jpg}
\end{textblock*}}

\makeatletter
\patchcmd{\beamer@sectionintoc}{\vskip0.8em}{\vskip0.8em}{}{}
\makeatother

%%========================================================================


    \title[]{Development of h-p Adjoint-based error estimation for LES of reactive flows }
	%\author[]{Christopher Ngigi}
	\author[]{{Christopher Ngigi\\ \tiny Ph.D. Pre-Candidate\\ \footnotesize Supervisor: Prof. C. P. T. Groth}}
	\titlegraphic{\includegraphics[width=0.15\textheight]{./figs/utias_logo_blue.jpg}}
	\institute[]{Doctoral Examination Committee \\ Meeting I \\ University of Toronto, Institute for Aerospace Studies}
	\date[]{April 6, 2015}
	%\usetheme{Warsaw}
	
%%========================================================================

\begin{document}

\addtocounter{framenumber}{-1}

\begingroup
\makeatletter
\setlength{\hoffset}{-.5\beamer@sidebarwidth}
\makeatother
\begin{frame}[plain]
    \titlepage	
\end{frame}




%==========================================================================
\begin{frame}[plain,c]
\begin{alertblock}{Outline: }
\tableofcontents
\end{alertblock}
\end{frame}

\endgroup

\section{Introduction}
\begin{frame}[allowframebreaks]

\frametitle{Introduction}
\begin{itemize}
\item Cost of experiment vs numerical simulation
\item Moore's law
\item Turbulent combustion
\end{itemize}
\end{frame}

%=========================================================================

\section[Scope]{Scope of research}
\begin{frame}[allowframebreaks]
\frametitle{Scope of research}
\begin{itemize}
\item Reducing numerical error
\item High Order ... CENO
\item Explicit filtering
\item Adjoint based error estimation
\item Using h and p adapatation
\end{itemize}
\end{frame}


%==========================================================================

\section[Methodology]{Methodology}
\begin{frame}[allowframebreaks]
\frametitle{Methodology}
\begin{itemize}
\item Favre Averaged Governing Equations
\item Large Eddy Simulation:
  \begin{itemize}
   \item Explicit Filtering
   \item Some LES errors: Aliasing, Commutation
   \item Sub-filter scale (SFS) modeling
  \end{itemize}
\item High-order finite volume methods: CENO technique - benefits of higher accuracy on a coarse mesh
\item AMR
   \begin{itemize}
   \item Block-based AMR: speed and parallelization
   \item Anisotropic vs Isotropic: how cell count (computational cost) can be reduced
   \item Now the non-uniform vs the uniform block modification
   \item Mesh geometry: CFFC can deal with cartesian or curvilinear coordinates - is this via using mapping functions for reference elements?
   \end{itemize}
\end{itemize}
\end{frame}

%==========================================================================
\section[Framework]{Existing framework}
\begin{frame}[allowframebreaks]
\frametitle{Existing framework}
\begin{itemize}
\item The CFFC code already includes the following required features:
\item Block-Based : people, year
\item AMR:
\item Deconick's research on explicit filters
\item High Order FVM with CENO:
\item Scott's work/input: Newton iterations and gmres solver
\item Lucie's non-uniform approach - improves accuracy of flux evaluations and reduces computational cost for anisotropic
\item PCM-FPI combustion modelling: modeled by F. Hernandez-Perez and N. Shahbazian
\item Initial adjoint analysis done by Martin for the advection equations
\end{itemize}
\end{frame}

%==========================================================================


\section[Error]{Overview of error}
\begin{frame}[allowframebreaks]
\frametitle{Overview of error}
Types of numerical error
\begin{itemize}
\item Truncation error
\item Solution error
Then explain a bit how they arise and how they can be dealt with
\end{itemize}
\end{frame}


%==========================================================================

\section[AMR]{Adaptive mesh refinement}
\begin{frame}[allowframebreaks]
\frametitle{Adaptive mesh refinement (AMR)}
\begin{itemize}
\item Benefits of AMR
\item How the block-based technique works
\item Ghost cells for intercommunication
\item Current stencils
\item how the high-order will affect the current stencil
\end{itemize}
\end{frame}





%==========================================================================

\section[FVM]{High Order CENO and FVM}
\begin{frame}[allowframebreaks]
\frametitle{High order finite volume method and CENO}
\begin{itemize}
\item Lucien's work
\item Ramy's work
\item Marc Charest's work
\item Luiz's work
\item how the high-order works, and how it reduces numerical error
\item separate slide of other groups researching this: Ihme and Poinssot. show some of their results
\end{itemize}
\end{frame}




%==========================================================================
\section[Adjoint]{Adjoint-based error estimation}

\begin{frame}[allowframebreaks]
\frametitle{Adjoint-based error estimation}
\begin{itemize}
\item separate slide on gradient-based techniques
\item separate slide explaining what the adjoint is. 
\begin{itemize}
\item who was the first to use adjoint
\item Cite initial work for this: Giles and Pierce, venditti and darmofal, fidkowski, jameson
\end{itemize}
\item continuous and discrete adjoint formulations
\begin{itemize}
\item continuous adjoint formulation
\item discrete adjoint formulation: methods to evaluate the discrete adjoint
\begin{itemize}
\item one
\item two
\item three
\end{itemize}
\end{itemize}
\item description of the adjoint methods to evaluate psi
\item techniques to evaluate dR/dU
\begin{itemize}
\item complex step
\item finite differencing
\item automated differentiation
\item approximate method
\end{itemize}
\item error estimation indicators, residual weighting (flag for refinement) and a 1D cartoon example, perhaps, of restriction/prolongation
\begin{itemize}
\item projecting onto fine space
\item restricting onto coarse space
\item getting the error in the residual and using this as a flag for refinement
\end{itemize}
\item treatment of steady vs unsteady adjoints
\item expected benefits of adjoint vs gradient based methods
\item separate slide on mesh adaptation as based on the adjoint. Enough diagrams from venditti and darmofal, fidkowski
\end{itemize}
\end{frame}


%==========================================================================
\section[Refinement]{Basis of refinement: h and p}
\begin{frame}[allowframebreaks]
\frametitle{Basis of refinement: h and p}
Show or put some figures with citations. Show what other groups have done. WHO has researched or is using \textbf{adjoint with AMR}?
\begin{itemize}
\item Fidkowski and Darmofal [2011] - Review of Output-Based Error Estimation and Mesh Adaptation in Computational Fluid Dynamics
\item Hartmann, ERROR ESTIMATION AND ADJOINT-BASED ADAPTATION IN AERODYNAMICS, [2006]
\item Nemec and Aftosmis [2007] - Adjoint Error Estimation and Adaptive Refinement for Embedded-Boundary Cartesian Meshes
\item Hartmann, Held and Leicht [2010] - Adjoint-based error estimation and adaptive mesh refinement for the RANS and k-ω turbulence model equations
\item Woopen, May and Sch{\"u}tz [2013] Adjoint-Based Error Estimation and Mesh Adaptation for Hybridized Discontinuous Galerkin Methods
\item Li, Allaneau and Jameson [2011] - Continuous Adjoint Approach for Adaptive Mesh Refinement
\item Diskin and Yamaleev [2011] Grid Adaptation Using Adjoint-Based Error Minimization
\end{itemize}
\end{frame}


%==========================================================================
\section[Usage]{How we can use this}

\begin{frame}[allowframebreaks]
\frametitle{How we can use this}
\begin{itemize}
\item using it for mesh refinement - how some previous groups used this
\item how we can link mesh adaptation AMR to the adjoint via h
\item how we can use p based refinement
\end{itemize}
\end{frame}


%==========================================================================
\section[Progress]{Progress to date}

\begin{frame}[allowframebreaks]
\frametitle{Progress to date}

\begin{itemize}
\item CFFC code familiarization : LES test case - on parallel clusters - SciNET. Job scheduling and post-processing results (tecplot)
\item creating and solving linear systems in parallel implementation - trilinos and MPI
\begin{itemize}
\item 2D Poisson problem
\item  3D Poisson problem
\end{itemize}
\item Preliminary work with the discrete adjoint - shockcube problem
\begin{itemize}
\item give the initial states, l and r
\item how the code was modified - multiblock and multiproc for uniform blocks
\item some results
\item work in progress
\begin{itemize}
\item boundary conditions
\item compare with other techniques to get dR/dU
\end{itemize}
\end{itemize}
\end{itemize}
\end{frame}


%==========================================================================
\section[Timeline]{Timeline}

\subsection[Present]{Work done to date}
\begin{frame}[allowframebreaks]
\frametitle{Timeline: April 2015  - January 2016}
\begin{itemize}
\item Put a table of what you have done till now
\end{itemize}
\end{frame}


\subsection[Future]{Future work}
\begin{frame}[allowframebreaks]
\frametitle{Projected milestones}
\begin{itemize}
\item Put a table of what you will do in the next steps
\end{itemize}
\end{frame}


%==========================================================================



%
%\begin{frame}[allowframebreaks]
%
%{\color{blue} Colourful Sub-Title}
%
%  \frametitle<presentation>{Further Reading}    
%  \begin{thebibliography}{10}    
%  \beamertemplatebookbibitems
%  \bibitem{Autor1990}
%    A.~Autor.
%    \newblock {\em Introduction to Giving Presentations}.
%    \newblock Klein-Verlag, 1990.
%  \beamertemplatearticlebibitems
%  \bibitem{Jemand2000}
%    S.~Jemand.
%    \newblock On this and that.
%    \newblock {\em Journal of This and That}, 2(1):50--100, 2000.
%  \end{thebibliography}
%
%
%\begin{itemize}
%\item First bullet level.
%\begin{itemize}
%\item Second bullet level.
%\begin{itemize}
%\item Third bullet level.
%\end{itemize}
%\end{itemize}
%\item First bullet level again.
%\end{itemize} 
%
%\end{frame}
%
%%=========================================================================
%
%
%\begin{frame}
%\frametitle{Frame With Figure}
%
%\begin{figure}[h!]
%\centering
%\includegraphics[width=0.4\textheight]{./figs/BFS_FLOW_DRIVER.jpg}
%\caption*{\tiny Typical Backward Facing Step Flow Configuration as an Example [Driver 1985]}
%\end{figure}
%
%\begin{itemize}
% \item Important bullet point.
%\end{itemize} 
%\end{frame}
%
%%==========================================================================
%
%\begin{frame}
%\frametitle{Frame With Table}
%
%\begin{itemize}
%\item Important bullet point.
%\end{itemize}
%
%\begin{tiny}
%\begin{table}[t!]
%%\begin{center}
%\caption*{{\tiny Re-Attachment Lengths Predicted by DES Turbulence Models}}
%\begin{tabular}{|c|c|c|c|}
%\hline
%\textbf{Solver Type} & \textbf{Pressure-Velocity Coupling} & \textbf{Pressure Interpolation} & \textbf{Re-Attachment Length (m)} \\ \hline
%\textit{Density Based} & & & 0.091\\
%\hline
%\textit{Pressure Based} & Simple & Standard & 0.148   \\
%\hline
%\textit{Pressure Based} & Simple & Second Order & 0.129  \\
%\hline
%\textit{Experimental} & & & 0.0795$\pm$1.27$\times$10$^{-4}$ [Driver 1985]\\
%\hline
%
%\end{tabular}
%%\end{center}
%\end{table}
%\end{tiny}
%
%\end{frame}

%==========================================================================



\begin{frame}
\begin{center}
{\huge Thank You For Your Attention!} \\
\vspace*{1.5cm}
{\Large Questions?}
\end{center}

\end{frame}


%==========================================================================

\appendix
\newcounter{finalframe}
\setcounter{finalframe}{\value{framenumber}}

\begin{frame}[allowframebreaks] 
\frametitle{References}
\begin{thebibliography}{1} %Using the bibtex package is recoended but this should allow you do the bibliograghy ad-hoc.
\begin{tiny}
\beamertemplatetextbibitems

\bibitem{Driver85}
Driver, D., and Seegmiller, L., 1985, "Features of a Reattaching Turbulent Shear Layer in Divergent Channel Flow," American Institute of Aeronautics and Astronautics, \textbf{23}(2) pp. 163-171.

\end{tiny}
\end{thebibliography}
\end{frame}

\begin{frame}
\frametitle{Backup Slide}

\begin{itemize}
\item Important backup slide point.
\end{itemize}

\end{frame}

\setcounter{framenumber}{\value{finalframe}}
\end{document}